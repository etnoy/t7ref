\documentclass[tikz]{standalone}
\usepackage{tikz}
\usetikzlibrary{shapes.geometric, arrows}
\begin{document}
\tikzstyle{startstop} = [rectangle, rounded corners, minimum width=3cm, minimum height=1cm,text centered, draw=black, fill=red!30]
\tikzstyle{io} = [trapezium, trapezium left angle=70, trapezium right angle=110, minimum width=3cm, minimum height=1cm, text centered, draw=black, fill=blue!30]
\tikzstyle{process} = [rectangle, minimum width=3cm, minimum height=1cm, text centered, draw=black, fill=orange!30]
\tikzstyle{decision} = [diamond, minimum width=3cm, minimum height=1cm, text centered, draw=black, fill=green!30]
\begin{tikzpicture}
\node (start) [startstop] {Start};
\node (compensation) [process,below= of start] {Compensation};
\node (closedloop) [process,below=of compensation] {Closed loop};
\node (purge) [process,below=of compensation] {Correction for purge};
\node (multiplicative) [process,below=of purge] {Multiplicative (long-term)
adaptation};
\node (additive) [process,below=of multiplicative] {Additive adaptation};
\node (starting) [process,below=of additive] {Starting fuel quantity};
\node (quantity) [process,below=of starting] {Fuel quantity per combustion};
\node (duration) [process,below=of quantity] {Injector opening duration};
\node (twice) [process,below=of duration] {Injection twice per combustion};
\node (voltage) [process,below=of twice] {Voltage-dependent correction};
\node (fuelcut) [process,below=of voltage] {Fuel cut};
\node (activation) [startstop,below=of fuelcut] {Activation};
\end{tikzpicture}

\begin{comment}

\begin{description}
    \item[Compensation] In case of a cold engine, shortly after starting, rapid
        load changes, knocking or high loads, the current
        value is multiplied by a compensation factor
    \item [Closed loop] The closed loop value is used as a multiplier. The
        value is then sent to box 4
    \item[Correction for purge] Multiply by the value for purge adaptation. The
        value is sent to box 5
    \item[Multiplicative adaptation (long term fuel trim)] The multiplicative adaptation value is used as a
        multiplier and the new value is sent to box 6
    \item[Additive adaptation] The additive adaptation value is added and the new
        value is sent to box 7
    \item[Starting fuel quantity] If the engine has not yet started, starting fuel is
        selected. The value is sent to box 8
    \item[Fuel quantity per combustion to be injected] The fuel quantity per combustion is the amount of
        petrol to be supplied to the engine. The value is sent
        to box 9
    \item[Injector opening duration] Converts the value to the time during which the
        injector must be open and the new value is sent to
        box 10
    \item[Injection twice per combustion] Injection takes place twice per combustion until the
        camshaft position has been found. Injection duration
        is divided by two. The value is sent to box 11
    \item[Voltage dependant needle lift duration added
        (battery correction)]
        Adds the injector time delay, which is voltage
        dependant. The value is sent to box 12
    \item[ Fuel cut] The value is sent to box 13 unless fuel cut is active
    \item[ Activation of injector] At a DETERMINED crank shaft angle, the
        microprocessor will control the transistor for the
        injector that is next in the firing order
\end{description}
\end{comment}

\end{document}


